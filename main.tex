\documentclass[12pt]{beamer}

\usepackage[T1]{fontenc}
\usepackage[utf8]{inputenc}
\usepackage[british]{babel}
\usepackage{mathtools}
\usepackage{amsthm}
\usepackage{libertine}
\usepackage[libertine]{newtxmath}
\usepackage[mathscr]{euscript}
\usepackage{tikz-cd}
\usetikzlibrary{decorations.markings}
\usepackage{float}
\usepackage[
  backend=biber,
  style=alphabetic,
  maxnames=10,
  maxalphanames=10]{biblatex}
\addbibresource{refs.bib}
\usepackage{hyperref}
\urlstyle{same}

\usefonttheme{serif}
\usetheme{Singapore}
\usecolortheme{rose}

\title[Categorías derivadas y geometría birracional]{Categorías derivadas y \\ geometría birracional}
\author{Pedro N\'{u}\~{n}ez}
\institute{Red de Doctorandos en Matemáticas UCM}
\date{22 de Marzo de 2022}

\makeatletter
\hypersetup{
  pdftitle={@title},
  colorlinks,
  linkcolor=[rgb]{0.2,0.2,0.6},
  citecolor=[rgb]{0.2,0.6,0.2},
  urlcolor=[rgb]{0.6,0.2,0.2}}
\makeatother

\begin{document}

\frame{\titlepage}

\begin{frame}
  \frametitle{Plan de la charla}
  \tableofcontents
\end{frame}

\section{Categorías derivadas}

\begin{frame}
  \frametitle{Homología relativa a un subespacio}
  Sequencia exacta de complejos produce sequencia exacta larga de homologías.
\end{frame}

\begin{frame}
  \frametitle{Categorías abelianas}
  Es una propiedad, no una estructura (importante por el contraste con las trianguladas).
  Definición.
  Ejemplo:~grupos abelianos.
  Contraejemplo:~conjuntos.
  Complejos y sequencias exactas.
\end{frame}

\begin{frame}
  \frametitle{Objetivo}
  Partiendo de una categoría abeliana (la de grupos abelianos, por tener un ejemplo concreto en mente), construir una categoría cuyos objetos sean complejos de cadenas y cuyos morfismos nos permitan identificar complejos con la misma homología.
\end{frame}

\begin{frame}
  \frametitle{Definición de la categoría derivada}
  Mediante propiedad universal.
\end{frame}

\begin{frame}
  \frametitle{Construcción:~primer paso}
  Identificar morfismos homotópicos.
\end{frame}

\begin{frame}
  \frametitle{Ejemplo:~primer paso no basta}
  \[ 0 \to \mathbb{Z} \to \mathbb{Z} \to \mathbb{Z}/2\mathbb{Z} \to 0 \]
\end{frame}

\begin{frame}
  \frametitle{Observación:~podemos detectar equivalencias débiles con conos}
\end{frame}

\begin{frame}
  \frametitle{Observación:~estructura triangulada}
\end{frame}

\begin{frame}
  \frametitle{Construcción:~segundo paso}
  Cociente de Verdier (localización con cálculo de fracciones).
\end{frame}

\begin{frame}
  \frametitle{Ejemplo:~functores derivados}
  Por qué hace falta derivarlos, compatibilidad con la estructura triangulada.
\end{frame}

\section{El Programa del Modelo Minimal}

\begin{frame}
  \frametitle{Objetivo:~clasificar variedades algebraicas}
  Clasificar módulo isomorfismo es difícil.
  En vez de ello, intentamos clasificar módulo equivalencia birracional (módulo tener abiertos de Zariski densos isomorfos).
\end{frame}

\begin{frame}
  \frametitle{Curvas:~único representante proyectivo y liso}
  Clasificación según el género, es decir, según el grado del divisor canónico.
\end{frame}

\begin{frame}
  \frametitle{Superficies:~demasiados posibles representantes proyectivos y lisos}
  Problema:~blow-up produce nuevo representante distinto pero también proyectivo y liso.
\end{frame}

\begin{frame}
  \frametitle{Superficies:~criterio de Castelnuovo}
  Las curvas contractibles se manifiestan como curvas con intersección negativa con el divisor canónico.
\end{frame}

\begin{frame}
  \frametitle{Superficies:~algoritmo del MMP}
  Contraer curvas negativas.
  Posibles resultados:
  \begin{enumerate}
    \item Contracción divisorial.
    \item Fibración de Mori.
  \end{enumerate}
\end{frame}

\section{Categorías derivadas de variedades}

\begin{frame}
  \frametitle{La categoría derivada de una variedad}
  Fibrados vectoriales no bastan, por eso consideramos haces coherentes.
\end{frame}

\begin{frame}
  \frametitle{Propiedades reflejadas en la categoría derivada}
  \begin{enumerate}
    \item Conexitud (en concreto, irreducibilidad).
    \item Dimensión.
    \item Anillo canónico.
  \end{enumerate}
\end{frame}

\begin{frame}
  \frametitle{Descomposiciones semiortogonales}
  Las ortogonales no son posibles.
\end{frame}

\begin{frame}
  \frametitle{Descomposición del espacio proyectivo}
\end{frame}

\begin{frame}
  \frametitle{Descomposición inducida por un blow-up}
\end{frame}

\section{Problemas abiertos}

\begin{frame}
  \frametitle{Hipótesis DK}
\end{frame}

\begin{frame}
  \frametitle{Descomposiciones en superficies}
\end{frame}

\section{Referencias}

\begin{frame}
  \frametitle{Gracias por vuestra atención!}
  \nocite{gm03}
  \nocite{huy06}
  \nocite{kaw17}
  \nocite{mat02}
  \printbibliography[heading=none]
\end{frame}

\end{document}
