\documentclass[12pt]{beamer}

\usepackage[T1]{fontenc}
\usepackage[utf8]{inputenc}
\usepackage[british]{babel}
\usepackage{mathtools}
\usepackage{amsthm}
\usepackage{libertine}
\usepackage[libertine]{newtxmath}
\usepackage[mathscr]{euscript}
\usepackage{tikz-cd}
\usetikzlibrary{decorations.markings}
\usepackage{float}
\usepackage[
  backend=biber,
  style=alphabetic,
  maxnames=10,
  maxalphanames=10]{biblatex}
\addbibresource{refs.bib}
\usepackage{hyperref}
\urlstyle{same}

\usefonttheme{serif}
\usetheme{Singapore}
\usecolortheme{rose}

\title[Categorías derivadas y geometría birracional]{Categorías derivadas y \\ geometría birracional}
\author{Pedro N\'{u}\~{n}ez}
\institute{Red de Doctorandos en Matemáticas UCM}
\date{22 de Marzo de 2022}

\makeatletter
\hypersetup{
  pdftitle={@title},
  colorlinks,
  linkcolor=[rgb]{0.2,0.2,0.6},
  citecolor=[rgb]{0.2,0.6,0.2},
  urlcolor=[rgb]{0.6,0.2,0.2}}
\makeatother

\begin{document}

\frame{\titlepage}

\section{Categorías derivadas}

\begin{frame}
  \frametitle{Motivación histórica}
  \begin{enumerate}
    \item Topología algebraica.
      \vspace{5mm}
      \pause

    \item Álgebra homológica.
      \vspace{5mm}
      \pause

   \item Categorías derivadas como herramienta. 
      \vspace{5mm}
      \pause

    \item Simetría especular homológica, hipótesis DK...
  \end{enumerate}
\end{frame}

\begin{frame}
  \frametitle{Motivación más concreta:~objetos}
  \begin{center}
    ``Complejos de cadenas bien, grupos de homología mal.'' \\
    \hfill ---Richard Thomas.
  \end{center}
  \pause
  \begin{itemize}
    \item Motivación conceptual (teorema de Whitehead):~la homología \textit{de las triangulaciones} de un espacio simplemente conexo determina el tipo de homotopía.
      \pause
    \item Motivación práctica:~$H^{i}(X) \neq \operatorname{Hom}(H_{i}(X),\mathbb{Z})$.
  \end{itemize}
\end{frame}

\begin{frame}
  \frametitle{Motivación más concreta:~morfismos}
  Un morfismo de complejos de cadenas se llama \textit{casi-isomorfismo} si induce isomorfismos en homología.
  \pause
  \begin{itemize}
    \item Motivación conceptual:~el teorema de Whitehead sugiere identificar triangulaciones casi-isomorfas.
      \pause
    \item Motivación práctica:~al final nos interesa la homología.
  \end{itemize}
\end{frame}

\begin{frame}
  \frametitle{Definición mediante propiedad universal}
  Sea $\mathbf{Ch}(\mathbf{Ab})$ la categoría de los complejos de cadenas de grupos abelianos.
  \pause
  La \textit{categoría derivada} de $\mathbf{Ab}$ es una categoría $\mathbf{D}(\mathbf{Ab})$ junto con un functor $Q \colon \mathbf{Ch}(\mathbf{Ab}) \to \mathbf{D}(\mathbf{Ab})$ tal que:
  \pause
  \begin{enumerate}
    \item Si $f \colon A^{\bullet} \to B^{\bullet}$ es un casi-isomorfismo, entonces $Q(f)$ es un isomorfismo.
      \pause
    \item Para todo functor $F \colon \mathbf{Ch}(\mathbf{Ab}) \to \mathbf{C}$ con la propiedad anterior existe un functor $G \colon \mathbf{D}(\mathbf{Ab}) \to \mathbf{C}$ único salvo isomorfismo tal que $F \cong G \circ Q$.
  \end{enumerate}
  \begin{center}
    \begin{tikzcd}[ampersand replacement=\&]
      \mathbf{Ch}(\mathbf{Ab}) \arrow[swap]{d}{Q} \arrow{r}{F} \& \mathbf{C} \\
      \mathbf{D}(\mathbf{Ab}) \arrow[dashed, swap, bend right=30]{ur}{\exists ! G} \&
    \end{tikzcd}
  \end{center}
\end{frame}

\begin{frame}
  \frametitle{Descripción explícita}
  \begin{itemize}
    \item Los objetos de $\mathbf{D}(\mathbf{Ab})$ son complejos de cadenas $A^{\bullet}$.
      \pause
    \item Los morfismos $A^{\bullet} \to B^{\bullet}$ en $\mathbf{D}(\mathbf{Ab})$ son clases de equivalencia de diagramas de la forma
      \begin{center}
        \begin{tikzcd}[ampersand replacement=\&]
          \& C^{\bullet} \arrow[swap]{dl}{\text{casi-iso}} \arrow{dr} \& \\
          A^{\bullet} \& \& B^{\bullet},
        \end{tikzcd}
      \end{center}
      \pause
      y dos diagramas son equivalentes si están dominados por un tercer diagrama de tal forma que el diagrama resultante conmute módulo homotopía.
  \end{itemize}
\end{frame}

\begin{frame}
  \frametitle{Construcción (dos pasos)}
  \[ \mathbf{Ch}(\mathbf{Ab}) \xrightarrow{\text{Módulo homotopía}} \mathbf{K}(\mathbf{Ab}) \xrightarrow{\text{Invertir casi-isos}} \mathbf{D}(\mathbf{Ab}) \]
  \pause
  \begin{enumerate}
    \item El primer paso es natural y conveniente:
      \begin{itemize}
        \item Las equivalencias de homotopía son ejemplos habituales de casi-isomorfismos.
          \pause
        \item Trabajar módulo homotopía nos permite invertir casi-isomorfismos usando cálculo de fracciones.
      \end{itemize}
      \pause
    \item El segundo paso es necesario, no todo casi-isomorfismo es una equivalencia de homotopía.
      Por ejemplo:
      \[ \cdots \to 0 \to \mathbb{Z} \xrightarrow{2} \mathbb{Z} \to \mathbb{Z}/2\mathbb{Z} \to 0 \to \cdots \]
  \end{enumerate}
\end{frame}

\begin{frame}
  \frametitle{Estructura triangulada}
  \begin{itemize}
    \item $\mathbf{D}(\mathbf{Ab})$ sigue teniendo la propiedad de ser aditiva (y $Q$ es aditivo), pero no llega a tener la propiedad de ser abeliana.
      \pause

    \item Por tanto no podemos hablar de sequencias exactas en $\mathbf{D}(\mathbf{Ab})$.
      \pause
      Pero esta categoría admite una \textit{estructura triangulada}, que nos permite reemplazar las secuencias exactas por triángulos exactos
      \[ A^{\bullet} \xrightarrow{f} B^{\bullet} \to \operatorname{Cone}(f)^{\bullet} \to A^{\bullet}[1]. \]
  \end{itemize}
\end{frame}

\begin{frame}
  \frametitle{Motivación topológica de los triángulos}
  \begin{enumerate}
    \item Si $i \colon X \hookrightarrow Y$ es la inclusión de un buen subespacio, entonces la secuencia $X \hookrightarrow Y \to Y/X$ es coexacta.
      \pause
    \item $X \xrightarrow{f} Y \to \operatorname{Cone}(f)$ es coexacta para todo $f \colon X \to Y$.
      \pause
    \item Para $j \colon Y \hookrightarrow \operatorname{Cone}(f)$ se tiene $\operatorname{Cone}(j) \simeq \Sigma X$.
      \pause
    \item Iterando obtenemos la secuencia coexacta de Puppe:
      \[ X \xrightarrow{f} Y \xrightarrow{j} C(f) \xrightarrow{g} \Sigma X \xrightarrow{\Sigma f} \Sigma Y \xrightarrow{\Sigma j} \Sigma C(f) \xrightarrow{\Sigma g} \Sigma^{2} X \to \cdots \]
  \end{enumerate}
  Véase G.~E.~Bredon, \textit{Topology and Geometry}, \S VII.5.
\end{frame}

\begin{frame}
  \frametitle{Functores derivados}
  \begin{itemize}
    \item Un functor $F \colon \mathbf{Ab} \to \mathbf{A}$ que no sea exacto no puede inducir un functor $F \colon \mathbf{D}(\mathbf{Ab}) \to \mathbf{D}(\mathbf{A})$ de forma directa, no estaría bien definido.
      \pause
    \item Si $F \colon \mathbf{Ab} \to \mathbf{A}$ es exacto izquierdo, consideramos su \textit{functor derivado derecho} $\mathbf{R}F \colon \mathbf{D}^{+}(\mathbf{Ab}) \to \mathbf{D}^{+}(\mathbf{A})$.
      \pause
    \item Podemos calcular $\mathbf{R}F(A^{\bullet})$ como $F(I^{\bullet})$, donde $I^{\bullet}$ es un complejo de objetos inyectivos casi-isomorfo a $A^{\bullet}$.
      \pause
    \item Si $A$ es un grupo abeliano, podemos recuperar los functores derivados clásicos $R^{i}F(A)$ como
      \[ R^{i}F(A) = H^{i}(\mathbf{R}F(A)). \]
  \end{itemize}
\end{frame}

\section{Geometría birracional}

\begin{frame}
  \frametitle{Objetivo:~clasificar variedades algebraicas}
  \begin{itemize}
    \item Clasificar módulo isomorfismo es difícil.
      \pause
    \item Intentamos clasificar módulo \textit{equivalencia birracional} (módulo tener abiertos de Zariski densos isomorfos).
      \pause
    \item Por ejemplo, la esfera de Riemann y el plano complejo son birracionales.
  \end{itemize}
\end{frame}

\begin{frame}
  \frametitle{Curvas:~único representante proyectivo y liso}
  \begin{itemize}
    \item Dada una curva cualquiera, existe un único representante proyectivo y liso en su clase de equivalencia birracional.
  \end{itemize}
\end{frame}

\begin{frame}
  \frametitle{Superficies:~demasiados posibles representantes}
  \begin{itemize}
    \item Problema:~blow-up produce nuevo representante distinto pero también proyectivo y liso.
  \end{itemize}
\end{frame}

\begin{frame}
  \frametitle{Superficies:~algoritmo del MMP}
  \begin{enumerate}
    \item Input:~superficie proyectiva lisa $S$ con divisor canónico $K_{S}$.\label{input}
      \pause
    \item Hay alguna curva $C \subseteq S$ tal que $K_{S}\cdot C < 0$?
      \pause
      \begin{itemize}
        \item Sí:~podemos contraerla; pasamos al siguiente paso.
        \item No:~$S$ es un modelo minimal; FIN.
      \end{itemize}
      \pause
    \item Sea $\pi \colon S \to X$ la contracción de $C$.
      Es $\dim(X) < \dim(S)$?
      \pause
      \begin{itemize}
        \item Sí:~$\pi$ es una fibración de Mori; FIN.
        \item No:~$X$ es una superficie proyectiva lisa y $\pi$ es un blow-up; repetir paso \ref{input} con $X$.
      \end{itemize}
  \end{enumerate}
\end{frame}

\section{Categorías derivadas de variedades}

\begin{frame}
  \frametitle{La categoría derivada de una variedad}
  Sea $X$ una variedad proyectiva lisa.
  \begin{itemize}
    \item (Haces de secciones de) fibrados vectoriales sobre $X$ no forman una categoría abeliana, necesitamos considerar la categoría $\mathbf{Coh}(X)$ de haces coherentes sobre $X$.
      \pause
    \item $\mathbf{Coh}(X)$ es abeliana, y definimos $\mathbf{D}^{\mathrm{b}}(X) := \mathbf{D}^{\mathrm{b}}(\mathbf{Coh}(X))$.
      \pause
    \item \textbf{Meta-lema:}~todo haz coherente $\mathscr{F}$ admite una resolución finita $\mathscr{E}^{\bullet}$ formada por fibrados vectoriales, así que la mayoría de isomorfismos canónicos del álgebra lineal siguen siendo válidos en $\mathbf{D}^{\mathrm{b}}(X)$.
      \pause
    \item $\mathbf{D}^{\mathrm{b}}(X)$ detecta conexión, dimensión y anillo canónico.
  \end{itemize}
\end{frame}

\begin{frame}
  \frametitle{Descomposiciones semiortogonales}
  \begin{itemize}
    \item $\mathbf{D}^{\mathrm{b}}(X) = \langle \mathbf{A}, \mathbf{B} \rangle$ significa que no hay morfismos no nulos de $\mathbf{B}$ a $\mathbf{A}$ y que la subcategoría triangulada más pequeña de $\mathbf{D}^{\mathrm{b}}(X)$ que contiene a $\mathbf{A}$ y a $\mathbf{B}$ es $\mathbf{D}^{\mathrm{b}}(X)$.
      \pause
    \item \textbf{Idea:}~los morfismos del MMP se reflejan en forma de descomposiciones semiortogonales de $\mathbf{D}^{\mathrm{b}}(X)$.
  \end{itemize}
\end{frame}

\begin{frame}
  \frametitle{Ejemplo:~fibración de Mori}
  La fibración de Mori $\pi \colon \mathbb{P}^{2} \to \{ * \}$ induce una descomposición
  \[ \mathbf{D}^{\mathrm{b}}(\mathbb{P}^{2}) = \langle \mathscr{O}, \mathscr{O}(1), \mathscr{O}(2) \rangle. \]
  \pause

  La idea de la semiortogonalidad es
  \begin{align*}
    \operatorname{Hom}_{\mathbf{D}^{\mathrm{b}}(\mathbb{P}^{2})}(\mathscr{O}(2),\mathscr{O}[i]) & = \operatorname{Ext}^{i}(\mathscr{O},\mathscr{O}(-2)) \\
    & = H^{i}(\mathbf{R}\operatorname{Hom}(\mathscr{O},\mathscr{O}(-2))) \\
    & = H^{i}(\mathbf{R}\Gamma(\mathbb{P}^{2},\mathscr{O}(-2)) \\
    & = H^{i}(\mathbb{P}^{2},\mathscr{O}(-2)) = 0.
  \end{align*}
\end{frame}

\begin{frame}
  \frametitle{Ejemplo:~blow-up}
  El blow-up $\pi \colon \tilde{X} \to X$ de una superficie proyectiva lisa $X$ en un punto $p \in X$ induce una descomposición
  \[ \mathbf{D}^{\mathrm{b}}(\tilde{X}) = \langle \mathscr{O}_{E}(-E), \mathbf{D}^{\mathrm{b}}(X) \rangle. \]
\end{frame}

\section{Problemas abiertos}

\begin{frame}
  \frametitle{Hipótesis DK}
  \textbf{Conjetura:}~Sean $X$ e $Y$ dos variedades proyectivas lisas $K$-equivalentes, es decir, existe una variedad proyectiva lisa $Z$ y morfismos birracionales
  \begin{center}
    \begin{tikzcd}[ampersand replacement=\&]
      \& Z \arrow[swap]{dl}{f} \arrow{dr}{g} \& \\
      X \& \& Y
    \end{tikzcd}
  \end{center}
  tales que $f^{*}K_{X} \sim g^{*}K_{Y}$.
  \pause
  Entonces $X$ e $Y$ son $D$-equivalentes, es decir, existe una equivalencia triangulada $\mathbf{D}^{\mathrm{b}}(X) \cong \mathbf{D}^{\mathrm{b}}(Y)$.
\end{frame}

\begin{frame}
  \frametitle{Descomposiciones en superficies}
  \textbf{Pregunta:}~Sea $X$ una superficie proyectiva lisa minimal con $H^{0}(X,\mathscr{O}_{X}(K_{X})) \neq 0$.
  Es $\mathbf{D}^{\mathrm{b}}(X)$ indecomponible?
\end{frame}

\section{Referencias}

\begin{frame}
  \frametitle{Gracias por vuestra atención!}
  \nocite{gm03}
  \nocite{huy06}
  \nocite{kaw17}
  \nocite{mat02}
  \printbibliography[heading=none]
\end{frame}

\end{document}
